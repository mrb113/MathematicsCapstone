
\documentclass[a4paper,draft]{amsproc}
\usepackage{amssymb}
\usepackage{amscd} %% Package for commutative diagrams
%\usepackage[dvips]{graphicx} %% Package for inserting illustrations/figures

%% The following packages are useful (you may want to use them):
%\usepackage{refcheck} %% Checks whether enumerated equations are referred to or not.
                       %% Please remove unnecessary numbers.
%\usepackage{cmdtrack} %% Checks whether all author defined macros are used or not
                       %% (see the end of .log file); unused ones should be removed.
%% Both of the packages have some limitations---consult package documentations.

\theoremstyle{plain}
 \newtheorem{thm}{Theorem}[section]
 \newtheorem{prop}{Proposition}[section]
 \newtheorem{lem}{Lemma}[section]
 \newtheorem{cor}{Corollary}[section]
\theoremstyle{definition}
 \newtheorem{exm}{Example}[section]
 \newtheorem{dfn}{Definition}[section]
\theoremstyle{remark}
 \newtheorem{rem}{Remark}[section]
 \numberwithin{equation}{section}

%% Please, do not change the following four lines:
\renewcommand{\le}{\leqslant}\renewcommand{\leq}{\leqslant}
\renewcommand{\ge}{\geqslant}\renewcommand{\geq}{\geqslant}
\renewcommand{\setminus}{\smallsetminus}
\setlength{\textwidth}{28cc} \setlength{\textheight}{42cc}

\title[Matrix Actions]{Matrix Actions on Polynomials}

\author[Bergeron]{\bfseries Michelle Bergeron}

\address{
Department of Mathematics \\ % \hfill (Received 00 00 2010)\\
Case Western Reserve University   \\ %\hfill (Revised  00 00 2010)\\
Cleveland, OH\\
}
\email{mrb113@case.edu}

\begin{document}

\vspace{18mm} \setcounter{page}{1} \thispagestyle{empty}


\begin{abstract}
TODO: Once I write the paper, come up with an abstract to fit it. Matrices and polynomials and stuff. 
\end{abstract}

\maketitle

\section{Mathematical Foundations} 

TODO: I need some sort of introduction here! Math, yo. 

\subsection{Special linear groups of degree 2 mod n}
Special linear groups are a specific type of group that can be defined over any commutative ring with unity. We are interested in the case where the ring is $\mathbb{Z}/n$, the integers mod $n$. 
\begin{dfn}
The special linear group $SL_{2}$ mod $n$ is defined as all 2x2 matrices with determinant 1 comprised of entries mod $n$.
\end{dfn}

For example, $SL_{2}$ mod 2 is made up of the matrices:
$$
\begin{bmatrix}
 0&1 \\ 
 1&0 
\end{bmatrix}
\begin{bmatrix}
 0&1 \\ 
 1&1 
\end{bmatrix}
\begin{bmatrix}
 1&0 \\ 
 0&1 
\end{bmatrix}
\begin{bmatrix}
 1&0 \\ 
 1&1 
\end{bmatrix}
\begin{bmatrix}
 1&1 \\ 
 0&1 
\end{bmatrix}
\begin{bmatrix}
 1&1 \\ 
 1&0 
\end{bmatrix}
$$

\subsection{Polynomials with coefficients mod n}
We claim that the matrices in $SL_{2}$ mod $n$ can act on polynomials. To do that, we must define a specific set of polynomials to work with. 
We thus define the set of polynomials that the matrices: 
\begin{dfn}
Let $R[x,y]$ be the ring of polynomials in two variables, x and y, with coefficients in $R = \mathbb{Z}/n\mathbb{Z}$. An element of $R[x,y]$ of the form  $ax^{my^{n}}$, with $a$ in $R$, is called a monomial. Define the degree of the monomial $ax^{my^{n}}$ to be $m+n$. Say that an element $p$ in $R[x,y]$ is homogeneous if p is a sum of monomials all of the same degree. (Say that 0 is homogeneous of every degree, since $0=0x^{my^{n}}$ for all m and n.) Write $R[x,y]_{k}$ for the set of polynomials in $R[x,y]$ that are homogeneous of degree $k$.
\end{dfn}
For now, we will consider $R$ to be positive integers mod 2. That is, $R = \mathbb{Z}/2$. To construct $R[x,y]_{k}$, take $(x+y)^{k}$ and take the coefficients mod $n$.

\begin{prop}The number of elements in $R[x,y]_{k}$ is equal to $n^{k + 1}$, where $r$ is the number of elements in $R$. 
\end{prop}

\begin{proof}
We prove the proposition by induction: \\
\textbf{Inductive hypothesis: } TODO Actually do the proof on paper first. \\
\textbf{Basis step: } TODO \\
\textbf{Inductive step:} TODO \\

Therefore, the number of elements in $R[x,y]_{k}$is equal to $r^{k + 1}$. 
\end{proof}
To further illustrate these concepts, take the example below. 
\begin{exm}
\textbf{TODO} this might be a little confusing, too. 
Let $R = \mathbb{Z}/2$.  Let $n = 2$. Our goal is to generate $R[x,y]_{2}$, so we begin by expanding $(x + y)^{2}$. 
$$(x + y)^{2} = x^{2} + 2xy + y^{2} $$
This gives us an equation of the form $a*x^{2} + b*xy + c*y^{2} $

Next, we form all combinations of this equation by plugging the elements of $R$ into $a, b$, and $c$: 
\begin{align*}
0x^{2} + 0xy + 0y^{2} &= 0 \\
1x^{2} + 0xy + 0y^{2} &= x^{2} \\
0x^{2} +1xy + 0y^{2}  &= xy \\
0x^{2} + 0xy + 1y^{2} &= y^{2} \\
1x^{2} + 0xy + 1y^{2} &= x^{2} + y^{2} \\
1x^{2} + 1xy + 0y^{2} &= x^{2} + xy \\
0x^{2} + 1xy + 1y^{2} &= xy + y^{2} \\
1x^{2} + 1xy + 1y^{2}  &= x^{2} + xy + y^{2}\\
\end{align*}

Since  $R = \mathbb{Z}/2$, $r$ = 2. By the argument in Proposition 1.1, we can verify  in $R[x,y]_{2}$ is equal to $2^{2 + 1}$ = 8.
\end{exm}

\section{Matrix Actions on R[x,y]$_{k}$} 
So far, we have constructed special linear groups of matrices and sets of polynomials with coefficients mod n. This section details how the matrices in $SL_{2}$ mod $n$ can act on polynomials in $R[x,y]_{k}$. 

\subsection{Preliminaries}
For a polynomial in $R[x,y]_{k}$and a matrix in $SL_{2}$ mod $n$, the matrix acts on the polynomial according to the following rule, $\mathcal{A}$: 
$$
x\begin{bmatrix}
 a&b \\ 
 c&d 
\end{bmatrix} = xa + yb 
$$
$$
y\begin{bmatrix}
 a&b \\ 
 c&d 
\end{bmatrix} = xc + yd 
$$

However, most of the polynomials we are working with are more complex than simply $x$ and $y$. Therefore, we need to establish how to use the matrix action $\mathcal{A}$ on more complicated polynomials. 
\begin{dfn}
For $p, q \epsilon R[x,y]$ and $r \epsilon R$, require: 
\begin{enumerate}
\item $(p+q)\mathcal{A} = p\mathcal{A} + q\mathcal{A}$
\item $pq\mathcal{A} = (p\mathcal{A})(q\mathcal{A}$
\item $rp\mathcal{A} = r(p\mathcal{A})$
\end{enumerate}
\end{dfn}

\begin{exm}
\textbf{Problem:} 
Apply the matrix 
$\begin{bmatrix}
 0&1 \\ 
 1&0 
\end{bmatrix}$ in $SL_{2}$ mod $2$ to the polynomial $xy + y^{2}$ in $R[x,y]_{2}$ using $\mathcal{A}.$ \\ \\
\textbf{Solution:} 
To take $(xy + y^{2})\mathcal{A}$ using $\begin{bmatrix}
 0&1 \\ 
 1&0 
\end{bmatrix}$, we will first take $xy\mathcal{A} + y^{2}\mathcal{A}:$ 
First, compute $xy\mathcal{A}$ using the multiplication rule: \\
$$x\begin{bmatrix}
 0&1 \\ 
 1&0 
\end{bmatrix} = 0x + 1y = y$$

$$y\begin{bmatrix}
 0&1 \\ 
 1&0 
\end{bmatrix} = 1x + 0y = x$$

to get  $xy\mathcal{A} = (y*x)$ mod $2 = xy$. \\

Next, compute $y^{2}\mathcal{A}$: \\
We know that $y\mathcal{A} = x$, so we can multiply $(x*x)$ mod $2 = y^{2}$. 
Thus, we combine our results of $xy\mathcal{A} + y^{2}\mathcal{A}$ with the addition rule to get $(xy + y^{2})\mathcal{A} = x^{2} + xy$. 

\end{exm}

\subsection{Orbits}
When we allow every matrix in $SL_{2}$ mod $n$ to act on all polynomials in $R[x,y]_{k}$for fixed values of $n$ and $k$ , patterns begin to emerge.
One can observe that some polynomials in $R[x,y]_{k}$generate the same result after being acted on by every matrix in $SL_{2}$ mod $n$. 
\begin{exm}
Apply the matrices in $SL_{2}$ mod $2$ to the polynomials in $R[x,y]_{2}$ 
We demonstrate the process with $\begin{bmatrix}
 0&1 \\ 
 1&0 
\end{bmatrix} $, mapping $p \mapsto p\mathcal{A}$ for $p \epsilon$ $R[x,y]_{2}$: \\

%% TODO how to align
\begin{center}
$0 \rightarrow 0$, \\
$y^{2} \rightarrow x^{2}$, \\
$xy \rightarrow xy$, \\
$xy + y^{2} \rightarrow xy + x^{2}$, \\
$x^{2} \rightarrow y^{2}$, \\
$x^{2} + y^{2} \rightarrow x^{2} + y^{2}$, \\
$xy + x^{2} \rightarrow xy + y^{2}$, \\
$xy + x^{2} + y^{2} \rightarrow xy + x^{2} + y^{2}$
\end{center}

Performing this on the rest of the matrices in $SL_{2}$ mod $n$ are omitted.

\end{exm}

If one continues applying $\mathcal{A}$ to $R[x,y]_{2}$ using the rest of the matrices in $SL_{2}$ mod $2$, we see that two polynomials in $R[x,y]_{2}$ are always mapped to themselves: 0 and $x^{2} + xy + y^{2}$. Upon closer inspection, the terms fall into three groups: $\{0\}, \{x^{2}, y^{2}, x^{2} + y^{2}\}$ and $\{x^{2} + xy + y^{2}\}$. The terms within each group always map to each other. We call these groupings \textit{\textbf{orbits}}. 

\section{Irreducible Polynomial Sets}
To further understand the properties of these polynomial sets, we introduce the notion of irreducibility. 
First, we define alternate notation for $R[x,y]_{k}$ to make it easier to use to use with irreducible polynomial sets. 
\begin{dfn}
$R[x,y]_{k}$ = $\underline{k+1}$
\end{dfn} 
What does it mean for such a set to be irreducible? 
\begin{dfn}
A set $\underline{k}$ of the form $R[x,y]_{k-1}$ is irreducible if the only $SL_{2}(R)$ submodules are  $\underline{k}$ and $\{0\}$. 
\end{dfn} 

\begin{dfn}
$U$ $\subset$ $R$ is a  $SL_{2}(R)$ submodule of $\underline{k}$ if the following two criteria hold: 

\begin{enumerate}
  \item $p  \mathcal{A} \epsilon U$ for $p$ $\epsilon$ $U$ and $\mathcal{A} \epsilon$ $SL_{2}(R)$ 
  \item $rp + sq$ $\epsilon$ $U$ for $r, s$ $\epsilon$ $R$; $p, q$ $\epsilon$ $U$
\end{enumerate}

\end{dfn}
Therefore, $R[x,y]_{2}$ = $\underline{1}$, $R[x,y]_{3}$ = $\underline{2}$, etc. Using this new notation, we can then illustrate an example of polynomial set irreducibility. 

\begin{exm}
In  $SL_{2}$ mod $2$, $\underline{1}$ and $\underline{2}$ are irreducible. We show this by
\textbf{TODO} work out this example
\end{exm}

\subsection{Reducible Polynomial Sets}
We have shown that there exist irreducible polynomial sets such as $\underline{1}$ and $\underline{2}$ in  $SL_{2}$ mod $2$. However, we can also illustrate that sets of higher degree can be condensed to irreducibles. 
\textbf{TODO} Please do go on
\begin{exm}
In  $SL_{2}$ mod $2$, we can find $\underline{1}$ and $\underline{2}$ within $\underline{3}$. 
\textbf{TODO} work out this example

Therefore, we write that $\underline{3} \cong \underline{1} \bigoplus \underline{2}$. 
\end{exm}

We can continue the search in polynomial sets of a higher order. 
\textbf{TODO} do that

\subsection{Generalization of irreducible behavior in $SL_{2}$ mod $2$ }
\textbf{TODO} show the pattern and do a proof of some sort
In the examples, it is clear that $\underline{1}$ and $\underline{2}$ are irreducible within $SL_{2}$ mod $2$. We have observed that in polynomial sets of higher powers, then we can find copies of $\underline{1}$ and $\underline{2}$ within each of them; therefore, such sets of higher powers are reducible under $SL_{2}$ mod $2$ . 

Observe that for an arbitrary set $\underline{w}$ under $SL_{2}$ mod $2$, the values of the copies it contains add up to $w$. For example, we show above that $\underline{3} \cong \underline{1} \bigoplus \underline{2}$. Note that 1 + 2 = 3. 

\subsection{Irreducibles in $SL_{2}$ mod $3$}
\textbf{TODO} Give a few examples, show the pattern and do a proof of some sort

\subsection{Irreducible behavior in $SL_{2}$ mod $n$}
\textbf{TODO} Make a generalization
\begin{prop}
 In $SL_{2}$ mod $n$, terms $\underline{1}$, ..., $\underline{n}$ are irreducible. 
\end{prop}
\begin{proof}
{TODO} Prove it!
\end{proof}

\section{Conclusion}
{TODO} Wrap it up
\bibliographystyle{amsplain}
%% Add bibliography if applicable

\end{document}

