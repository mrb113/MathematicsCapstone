
\documentclass[a4paper,draft]{amsproc}
\usepackage{amssymb}
\usepackage{amscd} %% Package for commutative diagrams
%\usepackage[dvips]{graphicx} %% Package for inserting illustrations/figures

%% The following packages are useful (you may want to use them):
%\usepackage{refcheck} %% Checks whether enumerated equations are referred to or not.
                       %% Please remove unnecessary numbers.
%\usepackage{cmdtrack} %% Checks whether all author defined macros are used or not
                       %% (see the end of .log file); unused ones should be removed.
%% Both of the packages have some limitations---consult package documentations.

\theoremstyle{plain}
 \newtheorem{thm}{Theorem}[section]
 \newtheorem{prop}{Proposition}[section]
 \newtheorem{lem}{Lemma}[section]
 \newtheorem{cor}{Corollary}[section]
\theoremstyle{definition}
 \newtheorem{exm}{Example}[section]
 \newtheorem{dfn}{Definition}[section]
\theoremstyle{remark}
 \newtheorem{rem}{Remark}[section]
 \numberwithin{equation}{section}

%% Please, do not change the following four lines:
\renewcommand{\le}{\leqslant}\renewcommand{\leq}{\leqslant}
\renewcommand{\ge}{\geqslant}\renewcommand{\geq}{\geqslant}
\renewcommand{\setminus}{\smallsetminus}
\setlength{\textwidth}{28cc} \setlength{\textheight}{42cc}

\title[Running title (header)]{TITLE}

\subjclass[2010]{Primary REQUIRED; Secondary OPTIONAL}

%% Please use the newest classification -- 2010
%% available at  http://msc2010.org/MSC-2010-server.html
%% and the newest amsproc.cls -- from 2009!!
%% Please, classify to the third level,
%% e.g., 26A and 26Axx are not satisfsctory.

\keywords{optional, but desirable}

\author[Surname]{\bfseries Name Surname}

\address{
Department of Mathematics \\ % \hfill (Received 00 00 2010)\\
Our University   \\ %\hfill (Revised  00 00 2010)\\
Town\\
Country}
\email{user@server}

%% OTHER AUTHOR(S):
%\author[]{}
%\address{ }
%\email{}

\thanks{Partially supported by ... } %% optional

\dedicatory{Communicated by }
%% We use this for communication information.
%% If you want do dedicate your paper to somebody, then please use \thanks{}

\begin{document}

%{\begin{flushleft}\baselineskip9pt\scriptsize
%PUBLICATIONS DE L'INSTITUT MATH\'EMATIQUE\newline
%Nouvelle s\'erie, tome 91(105) (2012), od--do \hfill DOI:
%\end{flushleft}}
\vspace{18mm} \setcounter{page}{1} \thispagestyle{empty}


\begin{abstract}
An abstract is OBLIGATORY!
Please do not use author defined macros in the abstract
and avoid references to anything in the paper,
since the abstract will be detached from the article.
\end{abstract}

\maketitle

\section{Section title}  %% Please avoid complicated formulas in titles

Insert your text here. If there are subsections, then you may use:

\subsection{Subsection title}
You may also use subsubsections,
but put a line or two of text between the subsection
and the subsubsection titles please.

Proclaims (theorems, propositions,...) should be
inserted as follows:

\begin{thm} \label{some label} % of course, label is optional
Statement of the theorem.
\end{thm}

Please, do not put a proclaim immediately following a subtitle of any level.
Write a line or two of text in between.

\begin{proof}
A proof. For displayed equations (formulas) you may use
\begin{equation}\label{eq:a1b}
e^{i\pi}=-1
\end{equation}
and similar \LaTeX\ constructions (align(ed), multline, gather(ed),\dots).
That way you may refer to \eqref{eq:a1b} in the subsequent text.
We strongly encourage the usage of this dynamic system of referencing
instead of a statical one, for example to (1.1).

If you do not refer to an equation, then you may write it as
\[
e^{i\pi}=-1
\]
(preferred) or
\begin{equation*}
e^{i\pi}=-1
\end{equation*}
In such a starred version the equation will not be enumerated.
If you want to use a distinctive tag to an equation,
you may do so in the following manner:
\begin{equation}\label{distinctive}
e^{i\pi}=-1
\tag{$*$}
\end{equation}
So you can refer to \eqref{distinctive}.

The symbol for the end of a proof will appear automatically.
\end{proof}

Formulas should be displayed \emph{only}
if they must be enumerated for a subsequent reference
or if they are too long or complicated.
Please \emph{do not} enumerate displayed formulas that are not referred to.

Send figures/illustrations as eps files (each figure/illustration in a separate file).

Figures can be inserted in the following way:

\begin{figure}[htb]
%\includegraphisc[width=99mm]{filename.eps}
\caption{}
\label{some label}
\end{figure}

A commutative diagram can be inserted as follows:
\[
\begin{CD}
TQ @>{\kappa}>> T^*Q \\
@V{g\diamond}VV  @VV{g\cdot}V \\
TQ @>>{\kappa}> T^*Q \\
\end{CD}
\]

Only \emph{standard} abbreviations for names of journals and other serials 
should be used in references. (see http://zbmath.org/journals/)


\bibliographystyle{amsplain}
\begin{thebibliography}{n} %% n is number of items, or the largest label

\bibitem{1}\label{some label - optional} A.\,U. Thor, (not Thor, A.U.!)
\emph{Title of paper},
J. Math. \textbf{99} (2008), 111--222.

\bibitem{2} A.\,U. Thor,
\emph{Title of paper},
in: E. Ditor (ed.), \emph{Title of Book}, Publisher, City, Year, 888--999.

\end{thebibliography}

\end{document}

%% To be filled in the journal office:

@author:
@affiliation:
@title:
@language: English
@pages:
@classification1:
@classification2:
@keywords:
@abstract:
@filename:
@EOI


