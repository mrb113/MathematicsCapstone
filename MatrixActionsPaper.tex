
\documentclass[a4paper,draft]{amsproc}
\usepackage{amssymb}
\usepackage{amscd} %% Package for commutative diagrams
%\usepackage[dvips]{graphicx} %% Package for inserting illustrations/figures

%% The following packages are useful (you may want to use them):
%\usepackage{refcheck} %% Checks whether enumerated equations are referred to or not.
                       %% Please remove unnecessary numbers.
%\usepackage{cmdtrack} %% Checks whether all author defined macros are used or not
                       %% (see the end of .log file); unused ones should be removed.
%% Both of the packages have some limitations---consult package documentations.

\theoremstyle{plain}
 \newtheorem{thm}{Theorem}[section]
 \newtheorem{prop}{Proposition}[section]
 \newtheorem{lem}{Lemma}[section]
 \newtheorem{cor}{Corollary}[section]
\theoremstyle{definition}
 \newtheorem{exm}{Example}[section]
 \newtheorem{dfn}{Definition}[section]
\theoremstyle{remark}
 \newtheorem{rem}{Remark}[section]
 \numberwithin{equation}{section}

%% Please, do not change the following four lines:
\renewcommand{\le}{\leqslant}\renewcommand{\leq}{\leqslant}
\renewcommand{\ge}{\geqslant}\renewcommand{\geq}{\geqslant}
\renewcommand{\setminus}{\smallsetminus}
\setlength{\textwidth}{28cc} \setlength{\textheight}{42cc}

\title[Matrix Actions]{Matrix Actions on Polynomials}

\author[Bergeron]{\bfseries Michelle Bergeron}

\address{
Department of Mathematics \\ % \hfill (Received 00 00 2010)\\
Case Western Reserve University   \\ %\hfill (Revised  00 00 2010)\\
Cleveland, OH\\
}
\email{mrb113@case.edu}

\begin{document}

\vspace{18mm} \setcounter{page}{1} \thispagestyle{empty}


\begin{abstract}
Using a matrix action, matrices in the group $SL_{2}$ mod $n$ can act on homogenous polynomials in two variables. We find homogenous (bivariate) polynomials of any given degree are closed under the matrix action, and that the mappings from a polynomial $p \in R[x,y]_{k}$ to  $p\mathcal{A}$ lay the foundation for certain patterns that we describe. 
\end{abstract}

\maketitle

\section{Mathematical Foundations} 

TODO: I need some sort of introduction here! Math, yo. 

\subsection{Special linear groups of degree 2 mod n}
Special linear groups are a specific type of group that can be defined over any commutative ring with unity. We are interested in the case where the ring is $\mathbb{Z}/n$, the integers mod $n$. 
\begin{dfn}
The special linear group $SL_{2}$ mod $n$ is defined as all 2x2 matrices with determinant 1 comprised of entries mod $n$.
\end{dfn}

For example, $SL_{2}$ mod 2 is made up of the matrices:
$$
\begin{bmatrix}
 0&1 \\ 
 1&0 
\end{bmatrix}
\begin{bmatrix}
 0&1 \\ 
 1&1 
\end{bmatrix}
\begin{bmatrix}
 1&0 \\ 
 0&1 
\end{bmatrix}
\begin{bmatrix}
 1&0 \\ 
 1&1 
\end{bmatrix}
\begin{bmatrix}
 1&1 \\ 
 0&1 
\end{bmatrix}
\begin{bmatrix}
 1&1 \\ 
 1&0 
\end{bmatrix}
$$

\subsection{Polynomials with coefficients mod n}
We claim that the matrices in $SL_{2}$ mod $n$ can act on polynomials. To do that, we must define a specific set of polynomials to work with. 
We thus define the set of polynomials that the matrices act on: 
\begin{dfn}
Let $R[x,y]$ be the ring of polynomials in two variables, x and y, with coefficients in $R = \mathbb{Z}/n\mathbb{Z}$. An element of $R[x,y]$ of the form  $ax^{m}y^{n}$, with $a$ in $R$, is called a monomial. Define the degree of the monomial $ax^{m}y^{n}$ to be $m+n$. Say that an element $p$ in $R[x,y]$ is homogeneous if p is a sum of monomials all of the same degree. (Say that 0 is homogeneous of every degree, since $0=0x^{m}y^{n}$ for all m and n.) Write $R[x,y]_{k}$ for the set of polynomials in $R[x,y]$ that are homogeneous of degree $k$.
\end{dfn}
For now, we will consider $R$ to be positive integers mod 2. That is, $R = \mathbb{Z}/2$. The monomials of degree $k$ in $R[x,y]$ are $x^k, x^{k-1}y, \cdots, xy^{k-1}, y$.These monomials span $R[x,y]_{k}$ as an R-module. This means that any element of $R[x,y]_{k}$ can be written as $\sum_{j=0}^k a_jx^{k-j}y^j$ for some $a_{j}$ in $R$. 

Note that the number of elements in $R[x,y]_{k}$ is equal to $n^{k + 1}$. This is because for each polynomial, there are $n$ options for each coefficient and k + 1 terms.
 
To further illustrate these concepts, take the example below. 
\begin{exm}
Let $R = \mathbb{Z}/2$.  Let $n = 2$. Our goal is to generate $R[x,y]_{2}$. 
According to the above, every element in $R[x,y]_{2}$ has the form $a*x^{2} + b*xy + c*y^{2}$ for some $a, b, c$ $\in$  $R$.

We form all combinations of this equation by plugging the elements of $R$ into $a, b$, and $c$: 
\begin{align*}
0x^{2} + 0xy + 0y^{2} &= 0 \\
1x^{2} + 0xy + 0y^{2} &= x^{2} \\
0x^{2} +1xy + 0y^{2}  &= xy \\
0x^{2} + 0xy + 1y^{2} &= y^{2} \\
1x^{2} + 0xy + 1y^{2} &= x^{2} + y^{2} \\
1x^{2} + 1xy + 0y^{2} &= x^{2} + xy \\
0x^{2} + 1xy + 1y^{2} &= xy + y^{2} \\
1x^{2} + 1xy + 1y^{2}  &= x^{2} + xy + y^{2}\\
\end{align*}

Since  $R = \mathbb{Z}/2$, $r$ = 2. By the argument in Proposition 1.1, we can verify  in $R[x,y]_{2}$ is equal to $2^{2 + 1}$ = 8.
\end{exm}

\section{Matrix Actions on R[x,y]$_{k}$} 
So far, we have constructed special linear groups of matrices and sets of polynomials with coefficients mod n. This section details how the matrices in $SL_{2}$ mod $n$ can act on polynomials in $R[x,y]_{k}$. 

\subsection{Preliminaries}
For a polynomial in $R[x,y]_{k}$ and a matrix in $SL_{2}$ mod $n$, the matrix acts on the polynomial according to the following rule, $\mathcal{A}$: 
$$
x\begin{bmatrix}
 a&b \\ 
 c&d 
\end{bmatrix} = xa + yb 
$$
$$
y\begin{bmatrix}
 a&b \\ 
 c&d 
\end{bmatrix} = xc + yd 
$$

However, most of the polynomials we are working with are more complex than simply $x$ and $y$. Therefore, we need to establish how to use the matrix action $\mathcal{A}$ on more complicated polynomials. 
\begin{dfn}
For $p, q$ $\in$ $R[x,y]$ and $r$ $\in$ $R$, require: 
\begin{enumerate}
\item $(p+q)\mathcal{A} = p\mathcal{A} + q\mathcal{A}$
\item $pq\mathcal{A} = (p\mathcal{A})(q\mathcal{A}$
\item $rp\mathcal{A} = r(p\mathcal{A})$
\end{enumerate}
\end{dfn}

\begin{exm}
\textbf{Problem:} 
Apply the matrix 
$\begin{bmatrix}
 0&1 \\ 
 1&0 
\end{bmatrix}$ in $SL_{2}$ mod $2$ to the polynomial $xy + y^{2}$ in $R[x,y]_{2}$ using $\mathcal{A}.$ \\ \\
\textbf{Solution:} 
To take $(xy + y^{2})\mathcal{A}$ using $\begin{bmatrix}
 0&1 \\ 
 1&0 
\end{bmatrix}$, we will first take $xy\mathcal{A} + y^{2}\mathcal{A}:$ 
First, compute $xy\mathcal{A}$ using the multiplication rule: \\
$$x\begin{bmatrix}
 0&1 \\ 
 1&0 
\end{bmatrix} = 0x + 1y = y$$

$$y\begin{bmatrix}
 0&1 \\ 
 1&0 
\end{bmatrix} = 1x + 0y = x$$

to get  $xy\mathcal{A} = (y*x)$ mod $2 = xy$. \\

Next, compute $y^{2}\mathcal{A}$: \\
We know that $y\mathcal{A} = x$, so we can multiply $(x*x)$ mod $2 = y^{2}$. 
Thus, we combine our results of $xy\mathcal{A} + y^{2}\mathcal{A}$ with the addition rule to get $(xy + y^{2})\mathcal{A} = x^{2} + xy$. 

\end{exm}

\subsection{Orbits}
When we allow every matrix in $SL_{2}$ mod $n$ to act on all polynomials in $R[x,y]_{k}$for fixed values of $n$ and $k$ , patterns begin to emerge.
One can observe that some polynomials in $R[x,y]_{k}$generate the same result after being acted on by every matrix in $SL_{2}$ mod $n$. 
\begin{exm}
Apply the matrices in $SL_{2}$ mod $2$ to the polynomials in $R[x,y]_{2}$ 
We demonstrate the process with $\begin{bmatrix}
 0&1 \\ 
 1&0 
\end{bmatrix} $, mapping $p \mapsto p\mathcal{A}$ for $p$ $\in$ $R[x,y]_{2}$: \\

\begin{center}
$0 \rightarrow 0$, \\
$y^{2} \rightarrow x^{2}$, \\
$xy \rightarrow xy$, \\
$xy + y^{2} \rightarrow xy + x^{2}$, \\
$x^{2} \rightarrow y^{2}$, \\
$x^{2} + y^{2} \rightarrow x^{2} + y^{2}$, \\
$xy + x^{2} \rightarrow xy + y^{2}$, \\
$xy + x^{2} + y^{2} \rightarrow xy + x^{2} + y^{2}$
\end{center}

Performing this on the rest of the matrices in $SL_{2}$ mod $n$ are omitted.

\end{exm}

If one continues applying $\mathcal{A}$ to $R[x,y]_{2}$ using the rest of the matrices in $SL_{2}$ mod $2$, we see that two polynomials in $R[x,y]_{2}$ are always mapped to themselves: 0 and $x^{2} + xy + y^{2}$. Upon closer inspection, the terms fall into three groups: $\{0\}, \{x^{2}, y^{2}, x^{2} + y^{2}\}$ and $\{x^{2} + xy + y^{2}\}$. The terms within each group always map to each other. We call these groupings \textit{\textbf{orbits}}. 

\section{Irreducible Polynomial Sets}
To further understand the properties of these polynomial sets, we introduce the notion of irreducibility. 
First, we define alternate notation for $R[x,y]_{k}$ to make it easier to use to use with irreducible polynomial sets. 
\begin{dfn}
$R[x,y]_{k}$ = $\underline{k+1}$
\end{dfn} 
What does it mean for such a set to be irreducible? 
\begin{dfn}
A set $\underline{k}$ of the form $R[x,y]_{k-1}$ is irreducible if the only $SL_{2}(R)$ submodules are  $\underline{k}$ and $\{0\}$. 
\end{dfn} 

\begin{dfn}
$U$ $\subset$ $R$ is a  $SL_{2}(R)$ submodule of $\underline{k}$ if the following two criteria hold: 

\begin{enumerate}
  \item $p  \mathcal{A}$ $\in U$ for $p$ $\in$ $U$ and $\mathcal{A}$ $\in$ $SL_{2}(R)$ 
  \item $rp + sq$ $\in$ $U$ for $r, s$ $\in$ $R$; $p, q$ $\in$ $U$
\end{enumerate}

\end{dfn}
Therefore, $R[x,y]_{0}$ = $\underline{1}$, $R[x,y]_{1}$ = $\underline{2}$, etc. Using this new notation, we can then illustrate an example of polynomial set irreducibility. 

Note that in $SL_{2}$ mod $2$, there are two matrices which can generate the entire set by operating on each other, which we will call $S$ and $T$. In this case, $S$ = 
$\begin{bmatrix}
 0&1 \\ 
 1&0 
\end{bmatrix}$ and $T$ = $\begin{bmatrix}
 1&1 \\ 
 0&1 
\end{bmatrix}$. The reader is encouraged to verify that operations between these two matrices can generate the entirity of $SL_{2}$ mod $2$. Consequently, these generator matrices are all that are required to test irreducibility because the rest of the group can be derived from them. 

\begin{exm}
In  $SL_{2}$ mod $2$, $\underline{1}$ and $\underline{2}$ are irreducible.
To show that a set is irreducible, we need to show that the only submodules they contain is \{0\} and itself. \\

$\textbf{\underline{1}}$ = \{0, 1\}: 0 and \underline{1} are the trivial submodules. We will show that they are the only submodules by contradiction: Assume that \{1\} is a submodule. Take $0 \cdot 1$. By the definition of \textbf{TODO}, this should generate an element of \{1\}. However, $ 0 \cdot 1 = 0$. $0 \not\in \{1\}$. Contradiction! Therefore, the only submodules of $\underline{1}$ are \{0\} and itself. \underline{1} is hence irreducible. \qed \\

$\textbf{\underline{2}}$ = \{0, x, x + y, y\}. Once again, we need to show that the only two submodules are the trivial submodules, \{0\} and \underline{2}. Choose $V < \underline{2}$. If $x \in V$, then we know that $(x + y) \in V$ because $x\mathcal{T}$ = $(x + y)$. We also know that $y \in V$ because $x\mathcal{S} = y$. One can repeat this argument by beginning with $(x + y)$ and $y$ instead of with $x$, so $V$ must be equal to $\underline{2}$. Therefore, we conclude that the only submodules are \{0\} and $\underline{2}$. \qed

\end{exm}

\subsection{Reducible Polynomial Sets}
We have shown that there exist irreducible polynomial sets such as $\underline{1}$ and $\underline{2}$ in  $SL_{2}$ mod $2$. However, we can also illustrate that sets of higher degree can be condensed to irreducibles. If one looks closely enough, there are "copies" of $\underline{1}$ and $\underline{2}$ in $\underline{3}$, $\underline{4}$, and beyond. The best way to illustrate what this means is by example: 
\begin{exm}

In  $SL_{2}$ mod $2$, we can find $\underline{1}$ and $\underline{2}$ within $\underline{3}$. 
\textbf{TODO} work out this example

Therefore, we write that $\underline{3} \cong \underline{1} \bigoplus \underline{2}$. 
\end{exm}

By the same argument, one can verify that $\underline{4}$ contains two copies of $\underline{1}$ and one copy of $\underline{2}$. 

We can continue this process for the rest of $R[x,y]_{k}$. 

\subsection{Generalization of irreducible behavior in $SL_{2}$ mod $2$}
In the examples, it is clear that $\underline{1}$ and $\underline{2}$ are irreducible within $SL_{2}$ mod $2$. We have observed that in polynomial sets in $R[x,y]_{k}$  of higher powers (greater values of $k$), then we can find copies of $\underline{1}$ and $\underline{2}$ within each of them. Therefore, such sets of higher powers are reducible under $SL_{2}$ mod $2$ . 

Observe that for an arbitrary set $\underline{k}$ under $SL_{2}$ mod $2$, the values of the copies it contains add up to $k$. For example, we show above that $\underline{3} \cong \underline{1} \bigoplus \underline{2}$. Note that 1 + 2 = 3. 

As $k$ grows, we can see a pattern begin to emerge. We can examine our findings so far by arranging them in a table. 

\begin{table}[h]
    \begin{tabular}{|c|c|c|} 
	\hline
    Polynomial set & \# of $\underline{1}$ & \# of $\underline{2}$ \\ \hline
    $\underline{1}$ & 1 & 0 \\ \hline
    $\underline{2}$ & 0 & 1 \\ \hline
    $\underline{3}$ & 1 & 1 \\ \hline
    $\underline{4}$ & 2 & 1 \\ \hline
    $\underline{5}$ & 1 & 2 \\ \hline
    $\underline{6}$ & 2 & 2 \\ \hline
    $\underline{7}$ & 3 & 2 \\ \hline
    $\underline{8}$ & 2 & 3 \\ \hline
    $\underline{9}$ & 3 & 3 \\ \hline
    \end{tabular} \\
\begin{centering}$\vdots$\end{centering}
\end{table} 

In $SL_{2}$ mod $2$, to determine row $\underline{h}$'s contents, there is the following formula. Note that there is a pattern in the table periodic every three rows. Therefore, there are three different cases for each row within the pattern.
\begin{prop}
To determine row $h$'s contents, first take $h$ mod 3. \\
Case 1: $h$ mod $3 = 0$. Row contains: 
\begin{table}[h]
\centering
    \begin{tabular}{|l|l|}
    \hline
    \# of $\underline{1}$ & \# of $\underline{2}$ \\ \hline
    $h/3$     & $h/3$     \\ \hline
    \end{tabular}
\end{table}

Case 2: $h$ mod $3 = 1$. Row contains: 
\begin{table}[h]
\centering
    \begin{tabular}{|l|l|}
    \hline
    \# of $\underline{1}$ & \# of $\underline{2}$ \\ \hline
    $(h/3) + 1$     & $h/3$     \\ \hline
    \end{tabular}
\end{table}

Case 3: $h$ mod $3 = 2$. Row contains: 
\begin{table}[h]
\centering
    \begin{tabular}{|l|l|}
    \hline
    \# of $\underline{1}$ & \# of $\underline{2}$ \\ \hline
    $h/3$     & $(h/3) + 1$     \\ \hline
    \end{tabular}
\end{table}

\end{prop}

\subsection{Irreducibles in $SL_{2}$ mod $3$}
We can easily find the irreducible terms in $SL_{2}$ mod $3$. By applying the matrix action to all of the polynomials in  $R[x,y]_{3}$, see that the orbits \textbf{TODO: list orbits} exist. By the same argument used in $SL_{2}$ mod $2$, $\underline{1}, \underline{2}$ and $\underline{3}$ are irreducible. 

\subsection{Irreducible behavior in $SL_{2}$ mod $n$}
We have noticed that for the cases of $SL_{2}$ mod $n$ for $n = 2$ and $n = 3$, the sets $\underline{1}$, ..., $\underline{n}$ are irreducible. We posit that this is true for any $n$. 
\begin{prop}
 In $SL_{2}$ mod $n$, sets $\underline{1}$, ..., $\underline{n}$ are irreducible. 
\end{prop}
\textbf{TODO: How can we show this?}

\section{Conclusion}
When matrices in $SL_{2}$ mod $n$ can act on polynomials $R[x,y]_{k}$, patterns begin to emerge. When one examines polynomials of lower orders, distinctive mappings called orbits begin to appear. As the order increases, we find that polynomial sets contain copies of lower order sets.  In fact, $SL_{2}$ mod $n$, sets $\underline{1}$, ..., $\underline{n}$ are irreducible. We find that it is possible to find how many copies of the irreducibile sets each higher order set can be decomposed into and to discern what will happen as $k$ approaches infinity.
\bibliographystyle{amsplain}
%% Add bibliography if applicable

\end{document}

